\documentclass[12pt]{article}

\usepackage[spanish]{babel}
\usepackage[margin=1in]{geometry}
\usepackage{amsmath}
\usepackage{amssymb}

\newcommand{\dint}{\delta_{\text{int}}}
\newcommand{\dext}{\delta_{\text{ext}}}

\title{Título}
\author{D'Autilio Joel, Rossi Pablo}
\date{}

\begin{document}
\maketitle

\section{Especificación DEVS}

El DEVS que representa a una célula está definido como

\[ C = \langle X, Y, S, \dint, \dext, \lambda, ta \rangle \]

donde
\begin{itemize}
  \item $X = \{0,1\} \times \{0 \dots 7\} \cup \{\text{(actualizar, 8)}\}$

    Desde el puerto 0 hasta el 7 se recibe el valor de los vecinos. El puerto 8 recibe la solicitud de actualizar.

  \item $Y = \{0,1\}$

    Tiene una única salida que indica el estado de la célula.

  \item $S = \{0,1\} \times Bool \times V \times V \times \mathbb{R}_0^+$

    El estado de la célula es una tupla $(s, b, va, vn, \sigma)$ donde
    \begin{itemize}
      \item $s \in \{0,1\}$ es el estado actual de la célula.
      \item $b \in Bool$ indica si el vecindario cambió en el último paso.
      \item $va \in V$ (\textit{vecinos anteriores}) es el estado del vecindario en el último paso.
      \item $vn \in V$ (\textit{vecinos nuevos}) es el estado del vecindario actual.
      \item $\sigma \in \mathbb{R}_0^+$ es el tiempo de la próxima actualización.
    \end{itemize}

  \item $\dext((s, b, va, vn, \sigma), e, (x,p)) = \begin{cases}
    (s, True, va, vn', \sigma) & \text{si } p \in \{0\dots 7\} \\
    (s, b, va, vn, \sigma)     & \text{si } b = False \\
    (s', False, vn, vn, 0) & \text{si } p = 8 \text{ y } b = True
    \end{cases}$

    $vn' = \text{modificar}(vn, p, x)$

    $s' = \text{calcular}(s, va)$

  \item $\dint((s, b, va, vn, \sigma)) = (s, b, va, vn, \infty)$

  \item $\lambda((s, b, va, vn, \sigma)) = s$

  \item $ta((s, b, va, vn, \sigma)) = \sigma$
\end{itemize}


\end{document}
