\documentclass[12pt]{article}

\usepackage[spanish]{babel}
\usepackage[margin=1in]{geometry}
\usepackage{amsmath}
\usepackage{amssymb}

\newcommand{\dint}{\delta_{\text{int}}}
\newcommand{\dext}{\delta_{\text{ext}}}

\title{Título}
\author{D'Autilio Joel, Rossi Pablo}
\date{}

\begin{document}
\maketitle

\section{Especificación DEVS}

El DEVS que representa a una célula está definido como

\[ C = \langle X, Y, S, \dint, \dext, \lambda, ta \rangle \]

donde
\begin{itemize}
  \item $X = \{0, 1\} \times \{0, \dots, 8\}$

    La entrada por el puerto 8 es un par $(b, 8)$ donde $b$ es 0 o 1, e indica qué acción tomará la célula.

    La entrada por los puertos 0 a 7 es un par $(b, i)$ donde $b$ es 0 o 1, e indica si la $i$-ésima célula vecina está viva o muerta.

  \item $Y = \{0, 1\}$

    La salida es 1 si la célula está viva, 0 si está muerta.

  \item $S = \{0, 1\} \times \{0, \dots, 8\} \times \mathbb{R}_0^+$

    El estado es un par $(b, n, \sigma)$ donde $b$ es 0 o 1, e indica si la célula está viva o muerta; $n$ es un entero entre 0 y 8, e indica la cantidad de células vecinas vivas; y $\sigma$ es un número real no negativo, e indica el tiempo que falta para que la célula cambie de estado.

  \item $\dint((b, n, \sigma)) = (b, n, \infty)$

    Una vez que la célula produce su salida, se queda esperando a la siguiente solicitud de acción.

  \item $\dext((b, n, \sigma)) = \begin{cases}
      (s, v', \sigma) & p \in \{0, \dots, 7\} \\
      (s, n, 0) & p = 1 \land x = 0 \\
      (s', n, \infty) & p = 1 \land x = 1
    \end{cases}$

    En el primer caso, al recibir una señal de una célula vecina, se actualiza la cantidad de células vecinas vivas.

    En el segundo caso, al recibir una señal de acción `0', se produce la salida del estado actual.

    En el tercer caso, al recibir una señal de acción `1', se actualiza el estado de la célula dependiendo del estado actual y de la cantidad de vecinas vivas.

  \item $\lambda((b, n, \sigma)) = b$

  \item $ta((b, n, \sigma)) = \sigma$
\end{itemize}


\end{document}
